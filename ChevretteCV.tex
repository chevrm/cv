\documentclass[11pt,a4paper,roman]{moderncv}
% font size ('10pt', '11pt' and '12pt')
% paper size ('a4paper', 'letterpaper', 'a5paper', 'legalpaper', 'executivepaper' and 'landscape')
% font family ('sans' and 'roman')

% moderncv themes
\moderncvstyle{banking}
% style options are 'casual' (default), 'classic', 'oldstyle' and 'banking'
\moderncvcolor{black}
% color options 'blue' (default), 'orange', 'green', 'red', 'purple', 'grey' and 'black'

%\renewcommand{\familydefault}{\sfdefault}
% to set the default font; use '\sfdefault' for the default sans serif font, '\rmdefault' for the default roman one, or any tex font name

%\nopagenumbers{}
% uncomment to suppress automatic page numbering for CVs longer than one page

% character encoding
%\usepackage[utf8]{inputenc}
% if you are not using xelatex ou lualatex, replace by the encoding you are using

% adjust the page margins
\usepackage[scale=0.75]{geometry}

%\setlength{\hintscolumnwidth}{3cm}
% if you want to change the width of the column with the dates

%\setlength{\makecvtitlenamewidth}{10cm}
% for the 'classic' style, if you want to force the width allocated to your name and avoid line breaks. be careful though, the length is normally calculated to avoid any overlap with your personal info; use this at your own typographical risks...

% personal data
\name{Marc G.}{Chevrette}

%\title{ }
% optional, remove / comment the line if not wanted

\address{1123 East Dayton Ave., Unit A}{}{Madison, WI 53703}
\phone[fixed]{(401)~269~9173}
% "type" of the phone can be "mobile" (default), "fixed" or "fax"

\email{chevrm@gmail.com}

%\homepage{chevrm.github.io}
%\social[linkedin]{www.linkedin.com/in/chevrette}
\social[twitter]{wildtypeMC}
%\social[github]{chevrm}

%\extrainfo{additional information}

%\photo[56pt][0.4pt]{qrcode.png}
% '64pt' is the height the picture must be resized to, 0.4pt is the thickness of the frame around it (put it to 0pt for no frame) and 'picture' is the name of the picture file

%\quote{Some quote}

% to show numerical labels in the bibliography (default is to show no labels); only useful if you make citations in your resume
%\makeatletter
%\renewcommand*{\bibliographyitemlabel}{\@biblabel{\arabic{enumiv}}}
%\makeatother
%\renewcommand*{\bibliographyitemlabel}{[\arabic{enumiv}]}% CONSIDER REPLACING THE ABOVE BY THIS

% bibliography with mutiple entries
%\usepackage{multibib}
%\newcites{book,misc}{{Books},{Others}}


\begin{document}

\makecvtitle
\vspace{-12mm}
%\section{Research Overview}
%\cvitem{}{Scientist with microbial genomics, secondary metabolite biosynthesis, DNA sequencing, and molecular biology background. Experienced in development of novel applications using computational biology, molecular biology, bioengineering, and genomic sequencing approaches.  Specific expertise and interest in microbial natural products biosynthesis, drug discovery, chemical ecology, and molecular engineering.}
\section{Education}
\cventry{}{Doctor of Philosophy (Ph.D.)}{University of Wisconsin-Madison, Madison, WI}{In progress}{Genetics}{NIH Chemistry-Biology Interface Predoctoral Fellow \newline \textbf{Advisor:} Cameron Currie, Ph.D.}{}
\cventry{}{Master of Liberal Arts (ALM)}{Harvard University Extension, Cambridge, MA}{03/2015}{Biotechnology -- Bioengineering \& Nanotechnology}{\textbf{Advisor:} Tom\'{a}s Maira-Litr\'{a}n, Pharm.D., Ph.D. \newline \textbf{Thesis:} Transposon-Directed Insertion Site Sequencing for Determination of Fitness Factors in Pulmonary Infection by \textit{Acinetobacter baumannii}.}{}
\cventry{}{Bachelor of Science}{Rensselaer Polytechnic Institute, Troy, NY}{12/2010}{Molecular Biology \& Bioinformatics}{}

\section{Research Experience}
%\subsection{Experience}
\cventry{Madison, WI}{Graduate Research Assistant}{Currie Lab, University of Wisconsin-Madison}{08/2015--present}{}{
\begin{itemize}
\item Built genomics-driven computational and analytical pipelines to uncover novel therapeutics and biosynthesis in free-living and host-associated actinomycetes.
\end{itemize}}
\cventry{Washington, DC (remote)}{Lead Computational Biologist}{Johnson Biosignatures Lab, Harvard \& Georgetown Universities}{10/2013--10/2015}{}{
\begin{itemize}
\item Performed whole genome sequencing and metagenomic analysis of environmental samples from sulfur-rich, extreme environments with implications in microbial ecology, biogeochemistry, and exobiology.
\item Characterized biosynthetic potential of metagenomic data.
\end{itemize}}
\cventry{Cambridge, MA}{Head of Experimental Genomics}{Warp Drive Bio}{04/2013--08/2015}{(04/2014--08/2015)}{}\vspace{-6mm}
\cventry{}{Research Associate, Experimental Genomics \& Computational Biology}{}{}{(04/2013--03/2014)}{ %\newline{}
\begin{itemize}
\item Executed genomic-directed natural products drug discovery, high throughput Next Generation Sequencing (htNGS), computational biology, and molecular biology of actinomycetes and fungi.
\item Designed and implemented genomic natural products searches over various scaffolds of business development and internal interest.
\item Developed and curated computational pipelines and databases for assembly, annotation, and custom analysis of public and internal htNGS data ($1.6\times10^{5}$ bacterial genomes, >150 closed and complete genomes) for analysis of novel polyketide, non-ribosomal peptide, and other natural product classes.
\item Handled processing and management of sequence data, predictions, and analyses supporting multiple projects across discovery, molecular biology, engineering, and synthetic biology.
\item Executed elucidation and prediction of novel chemical products of bacterial biosynthetic gene clusters and metabolic pathways (e.g. beta-lactams, aminoglycosides, rapamycin analogues, etc.).
\item Developed internal pipelines for applied phylogenomic annotations and prioritizations of multiple data types to inform discovery and engineering efforts.
\item Oversaw all lab and experimental support of actinomycete and fungal sequencing efforts for Illumina, Pacific Biosciences, and Oxford-Nanopore platforms.
\end{itemize}}
\vspace{-17mm} % necessary to back up to upper margin.  17 is arbitrary, if overshoot latex defaults to the margin
\cventry{}{}{}{}{}{
\begin{itemize}
\item Bioinformatics software development to support molecular and synthetic biology efforts.
\item Direct written and verbal communication of findings to
senior leadership and business partners.
\item Database management and delivery of sequence information to molecular biology, microbiology, and chemistry groups to aid drug discovery, strain engineering, and generation of expression constructs.
\end{itemize}}
%\clearpage % push next header to next page...
\cventry{Boston, MA}{Research Assistant, Microbiology \& Computational Biology}{Maira-Litr\'{a}n Infectious Disease Lab, Brigham \& Women's Hospital}{03/2013--08/2015}{}{
\begin{itemize}
\item Investigated \textit{in vivo} fitness, horizontal gene transmission, and pathogenesis of \textit{Acinetobacter baumannii}, \textit{Staphylococcus aureus}, \textit{Salmonella typhii}, and other virulent pathogens through microbiology, computational, and genomic techniques.
\item Developed and optimized genetic tools to enable novel examinations of pathogen fitness, invasion, and virulence using high-throughput transposon-directed insertion site sequencing of infections in murine models.
\end{itemize}}
\cventry{Cambridge, MA}{Research Associate II, Molecular Biology Process Development}{Broad Institute of MIT \& Harvard}{01/2011--03/2013}{}{
\begin{itemize}
\item Independently designed development initiatives including supporting htNGS, microfluidics, and automation goals.
\item Oversaw production and up-scaling of microbial mate-pair library construction (LC), integrated internal development with vendor technologies, and managed sample-tracking via real-time messaging to internal LIMS.
\item Increased throughput of microbial LC Platform 4-fold by automation and protocol development.
\item Worked extensively with mate-pair NGS LC, sequence analysis tools, genomic databases, statistical software, and programming/operating lab robotics.
\end{itemize}}
\cventry{Troy, NY}{Research Associate, Molecular Genetics}{Rutledge Molecular Genetics Lab, Rensselaer Polytechnic Institute}{05/2010--12/2010}{}{
\begin{itemize}
\item Designed and developed protocols and operating procedures for transgenic \textit{Caenorhabditis elegans} cultures to model stress-induced neural degeneration and Parkinson's Disease.
\end{itemize}}
\cventry{Jamestown, RI}{Research Assistant, Microbiology}{BCR Biotech}{09/2009--12/2009}{}{
\begin{itemize}
\item Wrote and optimized protocols and methods for engineering synthetic biosensing functions in \textit{Bacillus} spores.
\end{itemize}}
%\subsection{Miscellaneous}
%\cventry{year--year}{Job title}{Employer}{City}{}{Description}

\section{Publications}
\cvitem{P6}{Lewin, GR, M Davis, BR McDonald, E Glasgow, AJ Book, \textbf{MG Chevrette}, S Suh, A Boll, BG Fox, CR Currie.  \textit{In preparation}. "Long-term Cellulose Enrichment Selects for Highly Cellulytic Taxa and Competition for Public Goods."}
\cvitem{P5}{\textbf{Chevrette, MG}, F Aicheler, O Kohlbacher, CR Currie, MH Medema.  \textit{In preparation}. "Ensemble Computational Predictions of NRPS Adenylation Domain Substrate Specificities."}
\cvitem{P4}{Adnani, N, DR Braun, BR McDonald, \textbf{MG Chevrette}, CR Currie, TS Bugni.  \textit{In preparation}. "Complete Genome of \textit{Rhodococcus sp. strain WMMA-185}, a Marine Sponge-associated Bacterium."}
\cvitem{P3}{Horn, HA, E Gemperline, \textbf{MG Chevrette}, E Mevers, J Clardy, L Li, CR Currie.  \textit{In preparation}.  "Mass Spectrometry Imaging Reveals Differential Chemical Response to Pathogens in an Ancient Ant-Microbe Symbiosis."}
\cvitem{P2}{Lewin, GR, C Carlos, \textbf{MG Chevrette}, HA Horn, BR McDonald, RJ Stankey, BG Fox, CR Currie.  \textit{In press}.  "Ecology and Evolution of Actinobacteria and their Bioenergy Applications."} \vspace{1.5mm}
\cvitem{P1}{Johnson, SS, \textbf{MG Chevrette}, BL Ehlmann, KC Benison.  2015.  "Insights from the Metagenome of an Acid Salt Lake: the Role of Biology in an Extreme Depositional Environment."  \textit{PLOS ONE}.  2015 Apr; \textbf{10}(4):e0122869.}
\vspace{2.5mm}
%\cvitemwithcomment{Language 1}{Skill level}{Comment}
%\cvitemwithcomment{Language 2}{Skill level}{Comment}
%\cvitemwithcomment{Language 3}{Skill level}{Comment}

\section{Abstracts}
\cvitem{A15}{\textbf{\underline{Chevrette, MG}}, CR Currie, MH Medema.  prediCAT: An Accurate Predictive Method for Substrate Specificity of Nonribosomal Peptide Synthetase Adenylation Domains.  Presented at: 30th Annual Kenneth B. Raper Symposium on Microbial Research; Madison, WI; Sep 2, 2016.}
\cvitem{A14}{\underline{Bratburd, J}, BR McDonald, \textbf{Chevrette, MG}, JL Klassen, HA Horn, CR Currie.  Comparative Genomics of Fungus-growing Ant-associated Pseudonocardia.  Presented at: 30th Annual Kenneth B. Raper Symposium on Microbial Research; Madison, WI; Sep 2, 2016.}
\cvitem{A13}{\underline{Horn, HA}, E Gemperline, \textbf{MG Chevrette}, BR Mcdonald, J Bratburd, E Mevers, J Clardy, L Li, CR Currie.  Mass Spectrometry Imaging Reveals Differential Chemical Response to Pathogens in an Ancient Ant-Microbe Symbiosis.  Presented at: ISME International Symposium on Microbial Ecology; Montreal, QC, Canada; Aug 21-26, 2016.}
\cvitem{A12}{\textbf{\underline{Chevrette, MG}}, CR Currie, MH Medema.  Computational Predictions of Substrate Specificity in Nonribosomal Peptide Synthetases through Comparative Adenylation Domain Trees.  Presented at: American Society for Microbiology Microbe; Boston, MA; Jun 16-20, 2016.}
\cvitem{A11}{\underline{Johnson, SS}, ML Soni, DJ Collins, KC Benison, MR Mormile, \textbf{MG Chevrette}, BL Ehlmann.  Biosignatures in Mars Analog Acid Salt Lakes.  Presented at: USRA Biosignature, Preservation and Detection in Mars Analog Environments; Lake Tahoe, Nevada; May 16-19, 2016.}
\cvitem{A10}{\textbf{\underline{Chevrette, MG}}, C Carlson, C Thomas, TS Bugni, CR Currie.  Multifaceted Antibiotic Profiling across Actinomycete Chemical Ecology.  Presented at: Perlman Antibiotic Discovery and Development Symposium; Madison, WI; Apr 29, 2016.}
\cvitem{A9}{\underline{N Adnani}, S Adibhatla, E Vazquez-Rivera, GA Ellis, D Braun, \textbf{MG Chevrette}, BR McDonald, C Thompson, JS Piotrowski, Q Yu, L Li, CR Currie, TS Bugni.  Driving Production of Novel Natural Products through Marine Microbial Interspecies Interactions.  Presented at: Gordon Marine Natural Products; Ventura, CA; Mar 6-11, 2016.}
\cvitem{A8}{\textbf{\underline{Chevrette, MG}}, DW Udwary, CR Currie, SS Johnson.  Functional Classification and Secondary Metabolism of an Extreme Metagenome.  Presented at: 29th Annual Kenneth B. Raper Symposium on Microbial Research; Madison, WI; Sep 1, 2015.}
\cvitem{A7}{\textbf{\underline{Chevrette, MG}}, BL Ehlmann, KC Benison, SS Johnson.  Microbial Diversity and Biosynthetic Potential of an Extreme Sediment Metagenome.  Presented at: Gordon Applied and Environmental Microbiology; South Hadley, MA; Jul 12-17, 2015.}
\cvitem{A6}{\textbf{\underline{Chevrette, MG}}, M Vinacur, T Maira-Litr\'{a}n.  Transposon-Directed Insertion Site Sequencing Reveals \textit{in vivo} Fitness Factors in \textit{A. baumannii} Lung Infections.  Presented at: Boston Bacterial Meeting; Cambridge, MA; Jun 18-19, 2015.}
\cvitem{A5}{\underline{Udwary, DW}, K Robison, \textbf{MG Chevrette}, GL Verdine.  Lessons from Long Read Assembly of 100+ Actinomycete Genomes.  Presented at: Gordon Marine Natural Products; Ventura, CA; Mar 2-7, 2014.}
\cvitem{A4}{\underline{Robison, K}, DW Udwary, \textbf{MG Chevrette}, GL Verdine.  Long Read Assembly of >100 Actinomycete Genomes.  Presented at: Advances in Genome Biology \& Technology; Marco Island, FL; Feb 12-15, 2014.}
\cvitem{A3}{\underline{Young, S}, S Steelman, R Daza, \textbf{MG Chevrette}, R Lintner, S Gnerre, A Berlin, B Walker, C Nusbaum, R Nicol.  Generation of High-quality Draft Assemblies with a Single Sequencing Library.  Presented at: Sequencing, Finishing, Analysis in the Future; Santa Fe, NM; May 29-31, 2013.} \vspace{1.5mm}
\cvitem{A2}{	\underline{Steelman, S}, R Daza, \textbf{MG Chevrette}, P Kompella, P Trang, T Surabian, R Lintner, CZ Zhang, J Jung, M Meyerson, C Nusbaum, R Nicol.  Automated Low Input Mate-Pair Library Construction for Illumina Sequencing.  Presented at: Advances in Genome Biology \& Technology; Marco Island, FL; Feb 15-18, 2012.}
\cvitem{A1}{	Steelman, S, R Daza, \underline{\textbf{MG Chevrette}}, P Kompella, P Trang, T Surabian, R Lintner, R Nicol.  Microbial Mate-Pair Library Construction for De Novo Detection of Structural Rearrangements.  Presented at: Broad Institute Symposium; Boston, MA; Nov 7-8, 2011.}

%\section{Invited Talks}
%\cvitem{}{\textbf{Chevrette, MG}.  "Mining Actinomycetes for Natural Products."  University of Wisconsin, Madison, WI. Aug 25, 2014.}\vspace{1.5mm}


\section{Service \& Outreach}
\cventry{}{Open Bioinformatics Foundation -- Google Summer of Code -- antiSMASH}{Mentor}{03/2016--08/2016}{}{
\begin{itemize}
\item Provided both scientific and technical mentorship in the development of predictive models for ribosomally synthesised and post-translationally modified peptide (RiPP) biosynthesis in the widely-used genome-mining software suite, antiSMASH.
\end{itemize}}
\cventry{}{Computational Biology, Ecology, \& Evolution (ComBEE) -- UW-Madison}{Co-chair}{01/2016--present}{}{
\begin{itemize}
\item Coordinated and scheduled speakers, discussions, workshops, and meetings focused on molecular microbial ecology and evolution, computational biology, and data science.
\end{itemize}}
\cventry{}{Discovery Niche -- Wisconsin Institutes for Discovery}{Co-organizer}{10/2015--11/2015}{}{
\begin{itemize}
\item Planned, built, and maintained interactive public exhibits showcasing natural products drug discovery and bioenergy research for local Madison, Wisconsin community. 
\end{itemize}}
\cventry{}{Wisconsin Science Festival}{Volunteer}{10/2015}{}{}
\cventry{}{Long Now Foundation -- Revive \& Restore}{Open Genomics Advisor}{04/2014--10/2015}{}{
\begin{itemize}
\item Advised projects with Revive and Restore and Cofactor Genomics seeking to understand the genomics of the endangered and extremely bottlenecked black footed ferret and the extinct heath hen in an effort to reintroduce genetic diversity and aid in restoration of healthy wild populations.
\end{itemize}}
\cventry{}{Broad Institute of MIT \& Harvard}{Environmental, Health, and Safety Representative}{01/2011--03/2013}{}{
}


\section{Achievements \& Awards}
\cventry{}{University of Wisconsin-Madison}{Vilas Travel Grant}{2016}{}{}
\cventry{}{National Institutes of Health, NIGMS -- UW-Madison}{Chemistry-Biology Interface Predoctoral Fellowship}{06/2016-present}{}{}
\cventry{}{Harvard University Extension}{Dean's Academic Achievement Award}{03/2015}{}{}
%\cventry{}{Warp Drive Bio}{Core Value Award Nomination -- \textit{"Courageous: Uncompromising Science"}}{2014}{}{}
%\cventry{}{Warp Drive Bio}{Core Value Award Nomination -- \textit{"Unbounded: Reimagining the Possible"}}{2014}{}{}
\cventry{}{Broad Institute of MIT \& Harvard}{Featured Scientific Researcher -- \textit{"Who is Broad?"}}{01/2012}{}{}
\cventry{}{Rensselaer Polytechnic Institute}{Rensselaer Alumni Scholarship}{2004--2008}{}{}
\cventry{}{Rensselaer Polytechnic Institute}{Sal H. Alfiero Scholarship}{2004--2008}{}{}
\cventry{}{Rensselaer Polytechnic Institute}{Rhode Island State Scholarship}{2004--2008}{}{}
%\cvdoubleitem{category 1}{XXX, YYY, ZZZ}{category 4}{XXX, YYY, ZZZ}


\section{Professional Societies \& Groups}
\cventry{}{}{International Chemical Biology Society}{2016--present}{}{}
\vspace{-5mm}
\cventry{}{}{Natural Products Discovery and Bioengineering Network}{2016--present}{}{}
\vspace{-5mm}
\cventry{}{}{American Society for Microbiology}{2015--present}{}{}
\vspace{-5mm}
\cventry{}{}{Computational Biology, Ecology, \& Evolution (ComBEE) -- UW-Madison}{2015--present}{}{}
\vspace{-5mm}
\cventry{}{}{JF Crow Institute for the Study of Evolution}{2015--present}{}{}
\vspace{-5mm}
\cventry{}{}{Society for Industrial Microbiology and Biotechnology}{2014--present}{}{}
\vspace{-5mm}
\cventry{}{}{Laboratory Robotics Interest Group -- New England Chapter}{2011--2015}{}{}

%\section{Other Research}
%\cvitem{Acknowledged in Publication}{Grad, YH, et al.  (2012). Genomic epidemiology of the Escherichia coli O104:H4 outbreaks in Europe, 2011. Proceedings of the National Academy of Sciences, 109(8), 3065-3070.}

\end{document}